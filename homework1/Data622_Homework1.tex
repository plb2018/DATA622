\documentclass[]{article}
\usepackage{lmodern}
\usepackage{amssymb,amsmath}
\usepackage{ifxetex,ifluatex}
\usepackage{fixltx2e} % provides \textsubscript
\ifnum 0\ifxetex 1\fi\ifluatex 1\fi=0 % if pdftex
  \usepackage[T1]{fontenc}
  \usepackage[utf8]{inputenc}
\else % if luatex or xelatex
  \ifxetex
    \usepackage{mathspec}
  \else
    \usepackage{fontspec}
  \fi
  \defaultfontfeatures{Ligatures=TeX,Scale=MatchLowercase}
\fi
% use upquote if available, for straight quotes in verbatim environments
\IfFileExists{upquote.sty}{\usepackage{upquote}}{}
% use microtype if available
\IfFileExists{microtype.sty}{%
\usepackage[]{microtype}
\UseMicrotypeSet[protrusion]{basicmath} % disable protrusion for tt fonts
}{}
\PassOptionsToPackage{hyphens}{url} % url is loaded by hyperref
\usepackage[unicode=true]{hyperref}
\hypersetup{
            pdftitle={Data 2 - Homework \#1},
            pdfauthor={Paul Britton},
            pdfborder={0 0 0},
            breaklinks=true}
\urlstyle{same}  % don't use monospace font for urls
\usepackage[margin=1in]{geometry}
\usepackage{color}
\usepackage{fancyvrb}
\newcommand{\VerbBar}{|}
\newcommand{\VERB}{\Verb[commandchars=\\\{\}]}
\DefineVerbatimEnvironment{Highlighting}{Verbatim}{commandchars=\\\{\}}
% Add ',fontsize=\small' for more characters per line
\usepackage{framed}
\definecolor{shadecolor}{RGB}{248,248,248}
\newenvironment{Shaded}{\begin{snugshade}}{\end{snugshade}}
\newcommand{\KeywordTok}[1]{\textcolor[rgb]{0.13,0.29,0.53}{\textbf{#1}}}
\newcommand{\DataTypeTok}[1]{\textcolor[rgb]{0.13,0.29,0.53}{#1}}
\newcommand{\DecValTok}[1]{\textcolor[rgb]{0.00,0.00,0.81}{#1}}
\newcommand{\BaseNTok}[1]{\textcolor[rgb]{0.00,0.00,0.81}{#1}}
\newcommand{\FloatTok}[1]{\textcolor[rgb]{0.00,0.00,0.81}{#1}}
\newcommand{\ConstantTok}[1]{\textcolor[rgb]{0.00,0.00,0.00}{#1}}
\newcommand{\CharTok}[1]{\textcolor[rgb]{0.31,0.60,0.02}{#1}}
\newcommand{\SpecialCharTok}[1]{\textcolor[rgb]{0.00,0.00,0.00}{#1}}
\newcommand{\StringTok}[1]{\textcolor[rgb]{0.31,0.60,0.02}{#1}}
\newcommand{\VerbatimStringTok}[1]{\textcolor[rgb]{0.31,0.60,0.02}{#1}}
\newcommand{\SpecialStringTok}[1]{\textcolor[rgb]{0.31,0.60,0.02}{#1}}
\newcommand{\ImportTok}[1]{#1}
\newcommand{\CommentTok}[1]{\textcolor[rgb]{0.56,0.35,0.01}{\textit{#1}}}
\newcommand{\DocumentationTok}[1]{\textcolor[rgb]{0.56,0.35,0.01}{\textbf{\textit{#1}}}}
\newcommand{\AnnotationTok}[1]{\textcolor[rgb]{0.56,0.35,0.01}{\textbf{\textit{#1}}}}
\newcommand{\CommentVarTok}[1]{\textcolor[rgb]{0.56,0.35,0.01}{\textbf{\textit{#1}}}}
\newcommand{\OtherTok}[1]{\textcolor[rgb]{0.56,0.35,0.01}{#1}}
\newcommand{\FunctionTok}[1]{\textcolor[rgb]{0.00,0.00,0.00}{#1}}
\newcommand{\VariableTok}[1]{\textcolor[rgb]{0.00,0.00,0.00}{#1}}
\newcommand{\ControlFlowTok}[1]{\textcolor[rgb]{0.13,0.29,0.53}{\textbf{#1}}}
\newcommand{\OperatorTok}[1]{\textcolor[rgb]{0.81,0.36,0.00}{\textbf{#1}}}
\newcommand{\BuiltInTok}[1]{#1}
\newcommand{\ExtensionTok}[1]{#1}
\newcommand{\PreprocessorTok}[1]{\textcolor[rgb]{0.56,0.35,0.01}{\textit{#1}}}
\newcommand{\AttributeTok}[1]{\textcolor[rgb]{0.77,0.63,0.00}{#1}}
\newcommand{\RegionMarkerTok}[1]{#1}
\newcommand{\InformationTok}[1]{\textcolor[rgb]{0.56,0.35,0.01}{\textbf{\textit{#1}}}}
\newcommand{\WarningTok}[1]{\textcolor[rgb]{0.56,0.35,0.01}{\textbf{\textit{#1}}}}
\newcommand{\AlertTok}[1]{\textcolor[rgb]{0.94,0.16,0.16}{#1}}
\newcommand{\ErrorTok}[1]{\textcolor[rgb]{0.64,0.00,0.00}{\textbf{#1}}}
\newcommand{\NormalTok}[1]{#1}
\usepackage{longtable,booktabs}
% Fix footnotes in tables (requires footnote package)
\IfFileExists{footnote.sty}{\usepackage{footnote}\makesavenoteenv{long table}}{}
\usepackage{graphicx,grffile}
\makeatletter
\def\maxwidth{\ifdim\Gin@nat@width>\linewidth\linewidth\else\Gin@nat@width\fi}
\def\maxheight{\ifdim\Gin@nat@height>\textheight\textheight\else\Gin@nat@height\fi}
\makeatother
% Scale images if necessary, so that they will not overflow the page
% margins by default, and it is still possible to overwrite the defaults
% using explicit options in \includegraphics[width, height, ...]{}
\setkeys{Gin}{width=\maxwidth,height=\maxheight,keepaspectratio}
\IfFileExists{parskip.sty}{%
\usepackage{parskip}
}{% else
\setlength{\parindent}{0pt}
\setlength{\parskip}{6pt plus 2pt minus 1pt}
}
\setlength{\emergencystretch}{3em}  % prevent overfull lines
\providecommand{\tightlist}{%
  \setlength{\itemsep}{0pt}\setlength{\parskip}{0pt}}
\setcounter{secnumdepth}{0}
% Redefines (sub)paragraphs to behave more like sections
\ifx\paragraph\undefined\else
\let\oldparagraph\paragraph
\renewcommand{\paragraph}[1]{\oldparagraph{#1}\mbox{}}
\fi
\ifx\subparagraph\undefined\else
\let\oldsubparagraph\subparagraph
\renewcommand{\subparagraph}[1]{\oldsubparagraph{#1}\mbox{}}
\fi

% set default figure placement to htbp
\makeatletter
\def\fps@figure{htbp}
\makeatother


\title{Data 2 - Homework \#1}
\author{Paul Britton}
\date{2020-04-02}

\begin{document}
\maketitle

{
\setcounter{tocdepth}{2}
\tableofcontents
}
The rpubs version of this work can be found
\href{https://rpubs.com/plb_lttfer/593848}{here}, and source/data can be
found on github
\href{https://github.com/plb2018/DATA622/tree/master/homework1}{here}.

\begin{Shaded}
\begin{Highlighting}[]
\NormalTok{df <-}\StringTok{ }\KeywordTok{read.csv}\NormalTok{(}\StringTok{"https://raw.githubusercontent.com/plb2018/DATA622/master/homework1/data/prospecting_dataset.txt"}\NormalTok{, }\DataTypeTok{header=}\NormalTok{T)}
\end{Highlighting}
\end{Shaded}

\subsection{Question 1}\label{question-1}

\begin{Shaded}
\begin{Highlighting}[]
\KeywordTok{kable}\NormalTok{(df,}\DataTypeTok{caption=}\StringTok{"Prospecting Dataset"}\NormalTok{)}
\end{Highlighting}
\end{Shaded}

\begin{longtable}[]{@{}lllll@{}}
\caption{Prospecting Dataset}\tabularnewline
\toprule
age.group & networth & status & credit\_rating &
class.prospect\tabularnewline
\midrule
\endfirsthead
\toprule
age.group & networth & status & credit\_rating &
class.prospect\tabularnewline
\midrule
\endhead
youth & high & employed & fair & no\tabularnewline
youth & high & employed & excellent & no\tabularnewline
middle & high & employed & fair & yes\tabularnewline
senior & medium & employed & fair & yes\tabularnewline
senior & low & unemployed & fair & yes\tabularnewline
senior & low & unemployed & excellent & no\tabularnewline
middle & low & unemployed & excellent & yes\tabularnewline
youth & medium & employed & fair & no\tabularnewline
youth & low & unemployed & fair & yes\tabularnewline
senior & medium & unemployed & fair & yes\tabularnewline
youth & medium & unemployed & excellent & yes\tabularnewline
middle & medium & employed & excellent & yes\tabularnewline
middle & high & unemployed & fair & yes\tabularnewline
senior & medium & employed & excellent & no\tabularnewline
\bottomrule
\end{longtable}

You have been hired by a local electronics retailer and the above
dataset has been given to you. Manager Bayes Jr.9th wants to create a
spreadsheet to predict is a customer is likely prospect. To that end:

\begin{enumerate}
\def\labelenumi{\arabic{enumi}.}
\tightlist
\item
  Compute prior probabilities for the Prospect Yes/No
\item
  Compute the conditional probabilities
  \(P(age-group=youth|prospect=yes)\) and
  \(P(age-group=youth|prospect=no)\) where age-group is a predictor
  variable. Compute the conditional probabilities for each predictor
  variable, namely,\((age\_group,networth,status,credit\_rating)\)
\item
  Assuming the assumptions of Naive Bayes are met, compute the posterior
  probability \(P(prospect|X)\) where X is one of the predictor
  variable.
\end{enumerate}

\subsubsection{1 Compute prior probabilities for the Prospect
Yes/No}\label{compute-prior-probabilities-for-the-prospect-yesno}

\begin{Shaded}
\begin{Highlighting}[]
\CommentTok{#group the "yes" and "no" to simplify things}
\NormalTok{yes <-}\StringTok{ }\NormalTok{df[df}\OperatorTok{$}\NormalTok{class.prospect }\OperatorTok{==}\StringTok{ "yes"}\NormalTok{,]}
\NormalTok{no <-}\StringTok{ }\NormalTok{df[}\OperatorTok{!}\NormalTok{df}\OperatorTok{$}\NormalTok{class.prospect }\OperatorTok{==}\StringTok{ "yes"}\NormalTok{,] }

\CommentTok{#compute priors}
\NormalTok{p.yes <-}\StringTok{ }\KeywordTok{nrow}\NormalTok{(yes)}\OperatorTok{/}\KeywordTok{nrow}\NormalTok{(df)}
\NormalTok{p.no <-}\StringTok{ }\DecValTok{1}\OperatorTok{-}\NormalTok{p.yes}

\CommentTok{#format for output}
\NormalTok{probs <-}\StringTok{ }\KeywordTok{data.frame}\NormalTok{(}\KeywordTok{c}\NormalTok{(p.yes,p.no))}
\KeywordTok{row.names}\NormalTok{(probs) <-}\StringTok{ }\KeywordTok{c}\NormalTok{(}\StringTok{"Yes"}\NormalTok{,}\StringTok{"No"}\NormalTok{)}
\KeywordTok{colnames}\NormalTok{(probs) <-}\StringTok{ }\KeywordTok{c}\NormalTok{(}\StringTok{"Prior.Probs"}\NormalTok{)}

\CommentTok{#output}
\KeywordTok{kable}\NormalTok{(probs)}
\end{Highlighting}
\end{Shaded}

\begin{longtable}[]{@{}lr@{}}
\toprule
& Prior.Probs\tabularnewline
\midrule
\endhead
Yes & 0.6428571\tabularnewline
No & 0.3571429\tabularnewline
\bottomrule
\end{longtable}

\subsubsection{2 Compute the conditional
probabilities}\label{compute-the-conditional-probabilities}

Compute the conditional probabilities
\(P(age-group=youth|prospect=yes)\) and
\(P(age-group=youth|prospect=no)\) where age-group is a predictor
variable. Compute the conditional probabilities for each predictor
variable, namely,\((age\_group,networth,status,credit\_rating)\)

\begin{Shaded}
\begin{Highlighting}[]
\NormalTok{name <-}\StringTok{ }\KeywordTok{c}\NormalTok{(}\StringTok{"P(age-group=youth|prospect=yes)"}\NormalTok{,}
\StringTok{"P(age-group=middle|prospect=yes)"}\NormalTok{,}
\StringTok{"P(age-group=senior|prospect=yes)"}\NormalTok{,}
\StringTok{"P(age-group=youth|prospect=no)"}\NormalTok{,}
\StringTok{"P(age-group=middle|prospect=no)"}\NormalTok{,}
\StringTok{"P(age-group=senior|prospect=no)"}\NormalTok{,}
\StringTok{"P(networth=high|prospect=yes)"}\NormalTok{,}
\StringTok{"P(networth=low|prospect=yes)"}\NormalTok{,}
\StringTok{"P(networth=medium|prospect=yes)"}\NormalTok{,}
\StringTok{"P(networth=high|prospect=no)"}\NormalTok{,}
\StringTok{"P(networth=low|prospect=no)"}\NormalTok{,}
\StringTok{"P(networth=medium|prospect=no)"}\NormalTok{,}
\StringTok{"P(status=employed|prospect=yes)"}\NormalTok{,}
\StringTok{"P(status=employed|prospect=no)"}\NormalTok{,}
\StringTok{"P(status=unemployed|prospect=yes)"}\NormalTok{,}
\StringTok{"P(status=unemployed|prospect=no)"}\NormalTok{,}
\StringTok{"P(credit=fair|prospect=yes)"}\NormalTok{,}
\StringTok{"P(credit=excellent|prospect=no)"}\NormalTok{,}
\StringTok{"P(credit=fair|prospect=yes)"}\NormalTok{,}
\StringTok{"P(credit=excellent|prospect=no)"}\NormalTok{)}

\CommentTok{#manually compute all 20 priors}
\NormalTok{value <-}\StringTok{ }\KeywordTok{vector}\NormalTok{(}\DataTypeTok{mode =} \StringTok{"list"}\NormalTok{, }\DataTypeTok{length =} \KeywordTok{length}\NormalTok{(name))}

\NormalTok{value[}\DecValTok{1}\NormalTok{]  <-}\StringTok{ }\KeywordTok{sum}\NormalTok{(yes}\OperatorTok{$}\NormalTok{age.group }\OperatorTok{==}\StringTok{ "youth"}\NormalTok{)}\OperatorTok{/}\StringTok{ }\KeywordTok{nrow}\NormalTok{(yes)         }\CommentTok{#P(age-group=youth|prospect=yes)}
\NormalTok{value[}\DecValTok{2}\NormalTok{]  <-}\StringTok{ }\KeywordTok{sum}\NormalTok{(yes}\OperatorTok{$}\NormalTok{age.group }\OperatorTok{==}\StringTok{ "middle"}\NormalTok{)}\OperatorTok{/}\StringTok{ }\KeywordTok{nrow}\NormalTok{(yes)        }\CommentTok{#P(age-group=middle|prospect=yes)}
\NormalTok{value[}\DecValTok{3}\NormalTok{]  <-}\StringTok{ }\KeywordTok{sum}\NormalTok{(yes}\OperatorTok{$}\NormalTok{age.group }\OperatorTok{==}\StringTok{ "senior"}\NormalTok{)}\OperatorTok{/}\StringTok{ }\KeywordTok{nrow}\NormalTok{(yes)        }\CommentTok{#P(age-group=senior|prospect=yes)}
\NormalTok{value[}\DecValTok{4}\NormalTok{]  <-}\StringTok{ }\KeywordTok{sum}\NormalTok{(no}\OperatorTok{$}\NormalTok{age.group }\OperatorTok{==}\StringTok{ "youth"}\NormalTok{)}\OperatorTok{/}\StringTok{ }\KeywordTok{nrow}\NormalTok{(no)           }\CommentTok{#P(age-group=youth|prospect=no)",}
\NormalTok{value[}\DecValTok{5}\NormalTok{]  <-}\StringTok{ }\KeywordTok{sum}\NormalTok{(no}\OperatorTok{$}\NormalTok{age.group }\OperatorTok{==}\StringTok{ "middle"}\NormalTok{)}\OperatorTok{/}\StringTok{ }\KeywordTok{nrow}\NormalTok{(no)          }\CommentTok{#P(age-group=middle|prospect=no)"}
\NormalTok{value[}\DecValTok{6}\NormalTok{]  <-}\StringTok{ }\KeywordTok{sum}\NormalTok{(no}\OperatorTok{$}\NormalTok{age.group }\OperatorTok{==}\StringTok{ "senior"}\NormalTok{)}\OperatorTok{/}\StringTok{ }\KeywordTok{nrow}\NormalTok{(no)          }\CommentTok{#P(age-group=senior|prospect=no)}
\NormalTok{value[}\DecValTok{7}\NormalTok{]  <-}\StringTok{ }\KeywordTok{sum}\NormalTok{(yes}\OperatorTok{$}\NormalTok{networth }\OperatorTok{==}\StringTok{ "high"}\NormalTok{)}\OperatorTok{/}\StringTok{ }\KeywordTok{nrow}\NormalTok{(yes)           }\CommentTok{#P(networth=high|prospect=yes)}
\NormalTok{value[}\DecValTok{8}\NormalTok{]  <-}\StringTok{ }\KeywordTok{sum}\NormalTok{(yes}\OperatorTok{$}\NormalTok{networth }\OperatorTok{==}\StringTok{ "low"}\NormalTok{)}\OperatorTok{/}\StringTok{ }\KeywordTok{nrow}\NormalTok{(yes)            }\CommentTok{#P(networth=low|prospect=yes)}
\NormalTok{value[}\DecValTok{9}\NormalTok{]  <-}\StringTok{ }\KeywordTok{sum}\NormalTok{(yes}\OperatorTok{$}\NormalTok{networth }\OperatorTok{==}\StringTok{ "medium"}\NormalTok{)}\OperatorTok{/}\StringTok{ }\KeywordTok{nrow}\NormalTok{(yes)         }\CommentTok{#P(networth=medium|prospect=yes)}
\NormalTok{value[}\DecValTok{10}\NormalTok{] <-}\StringTok{ }\KeywordTok{sum}\NormalTok{(no}\OperatorTok{$}\NormalTok{networth }\OperatorTok{==}\StringTok{ "high"}\NormalTok{)}\OperatorTok{/}\StringTok{ }\KeywordTok{nrow}\NormalTok{(no)             }\CommentTok{#P(networth=high|prospect=no)}
\NormalTok{value[}\DecValTok{11}\NormalTok{] <-}\StringTok{ }\KeywordTok{sum}\NormalTok{(no}\OperatorTok{$}\NormalTok{networth }\OperatorTok{==}\StringTok{ "low"}\NormalTok{)}\OperatorTok{/}\StringTok{ }\KeywordTok{nrow}\NormalTok{(no)              }\CommentTok{#P(networth=low|prospect=no)}
\NormalTok{value[}\DecValTok{12}\NormalTok{] <-}\StringTok{ }\KeywordTok{sum}\NormalTok{(no}\OperatorTok{$}\NormalTok{networth }\OperatorTok{==}\StringTok{ "medium"}\NormalTok{)}\OperatorTok{/}\StringTok{ }\KeywordTok{nrow}\NormalTok{(no)           }\CommentTok{#P(networth=medium|prospect=no)}
\NormalTok{value[}\DecValTok{13}\NormalTok{] <-}\StringTok{ }\KeywordTok{sum}\NormalTok{(yes}\OperatorTok{$}\NormalTok{status }\OperatorTok{==}\StringTok{ "employed"}\NormalTok{)}\OperatorTok{/}\StringTok{ }\KeywordTok{nrow}\NormalTok{(yes)         }\CommentTok{#P(status=employed|prospect=yes)}
\NormalTok{value[}\DecValTok{14}\NormalTok{] <-}\StringTok{ }\KeywordTok{sum}\NormalTok{(no}\OperatorTok{$}\NormalTok{status }\OperatorTok{==}\StringTok{ "employed"}\NormalTok{)}\OperatorTok{/}\StringTok{ }\KeywordTok{nrow}\NormalTok{(no)           }\CommentTok{#P(status=employed|prospect=no)}
\NormalTok{value[}\DecValTok{15}\NormalTok{] <-}\StringTok{ }\KeywordTok{sum}\NormalTok{(yes}\OperatorTok{$}\NormalTok{status }\OperatorTok{==}\StringTok{ "unemployed"}\NormalTok{)}\OperatorTok{/}\StringTok{ }\KeywordTok{nrow}\NormalTok{(yes)       }\CommentTok{#P(status=unemployed|prospect=yes)}
\NormalTok{value[}\DecValTok{16}\NormalTok{] <-}\StringTok{ }\KeywordTok{sum}\NormalTok{(no}\OperatorTok{$}\NormalTok{status }\OperatorTok{==}\StringTok{ "unemployed"}\NormalTok{)}\OperatorTok{/}\StringTok{ }\KeywordTok{nrow}\NormalTok{(no)         }\CommentTok{#P(status=unemployed|prospect=no)}
\NormalTok{value[}\DecValTok{17}\NormalTok{] <-}\StringTok{ }\KeywordTok{sum}\NormalTok{(yes}\OperatorTok{$}\NormalTok{credit_rating }\OperatorTok{==}\StringTok{ "fair"}\NormalTok{)}\OperatorTok{/}\StringTok{ }\KeywordTok{nrow}\NormalTok{(yes)      }\CommentTok{#P(credit=fair|prospect=yes)}
\NormalTok{value[}\DecValTok{18}\NormalTok{] <-}\StringTok{ }\KeywordTok{sum}\NormalTok{(no}\OperatorTok{$}\NormalTok{credit_rating }\OperatorTok{==}\StringTok{ "fair"}\NormalTok{)}\OperatorTok{/}\StringTok{ }\KeywordTok{nrow}\NormalTok{(no)        }\CommentTok{#P(credit=excellent|prospect=no)}
\NormalTok{value[}\DecValTok{19}\NormalTok{] <-}\StringTok{ }\KeywordTok{sum}\NormalTok{(yes}\OperatorTok{$}\NormalTok{credit_rating }\OperatorTok{==}\StringTok{ "excellent"}\NormalTok{)}\OperatorTok{/}\StringTok{ }\KeywordTok{nrow}\NormalTok{(yes) }\CommentTok{#P(credit=fair|prospect=yes)}
\NormalTok{value[}\DecValTok{20}\NormalTok{] <-}\StringTok{ }\KeywordTok{sum}\NormalTok{(no}\OperatorTok{$}\NormalTok{credit_rating }\OperatorTok{==}\StringTok{ "excellent"}\NormalTok{)}\OperatorTok{/}\StringTok{ }\KeywordTok{nrow}\NormalTok{(no)   }\CommentTok{#P(credit=excellent|prospect=no)}

\CommentTok{#bind names and values}
\NormalTok{q1 <-}\StringTok{ }\KeywordTok{data.frame}\NormalTok{(}\KeywordTok{cbind}\NormalTok{(name,value))}

\CommentTok{#output as a table}
\KeywordTok{kable}\NormalTok{(q1)}
\end{Highlighting}
\end{Shaded}

\begin{longtable}[]{@{}ll@{}}
\toprule
name & value\tabularnewline
\midrule
\endhead
P(age-group=youth\textbar{}prospect=yes) &
0.222222222222222\tabularnewline
P(age-group=middle\textbar{}prospect=yes) &
0.444444444444444\tabularnewline
P(age-group=senior\textbar{}prospect=yes) &
0.333333333333333\tabularnewline
P(age-group=youth\textbar{}prospect=no) & 0.6\tabularnewline
P(age-group=middle\textbar{}prospect=no) & 0\tabularnewline
P(age-group=senior\textbar{}prospect=no) & 0.4\tabularnewline
P(networth=high\textbar{}prospect=yes) &
0.222222222222222\tabularnewline
P(networth=low\textbar{}prospect=yes) & 0.333333333333333\tabularnewline
P(networth=medium\textbar{}prospect=yes) &
0.444444444444444\tabularnewline
P(networth=high\textbar{}prospect=no) & 0.4\tabularnewline
P(networth=low\textbar{}prospect=no) & 0.2\tabularnewline
P(networth=medium\textbar{}prospect=no) & 0.4\tabularnewline
P(status=employed\textbar{}prospect=yes) &
0.333333333333333\tabularnewline
P(status=employed\textbar{}prospect=no) & 0.8\tabularnewline
P(status=unemployed\textbar{}prospect=yes) &
0.666666666666667\tabularnewline
P(status=unemployed\textbar{}prospect=no) & 0.2\tabularnewline
P(credit=fair\textbar{}prospect=yes) & 0.666666666666667\tabularnewline
P(credit=excellent\textbar{}prospect=no) & 0.4\tabularnewline
P(credit=fair\textbar{}prospect=yes) & 0.333333333333333\tabularnewline
P(credit=excellent\textbar{}prospect=no) & 0.6\tabularnewline
\bottomrule
\end{longtable}

\subsubsection{\texorpdfstring{3 Assuming the assumptions of Naive Bayes
are met, compute the posterior probability \(P(prospect|X)\) where X is
one of the predictor
variable.}{3 Assuming the assumptions of Naive Bayes are met, compute the posterior probability P(prospect\textbar{}X) where X is one of the predictor variable.}}\label{assuming-the-assumptions-of-naive-bayes-are-met-compute-the-posterior-probability-pprospectx-where-x-is-one-of-the-predictor-variable.}

It appears as though the assumptions of naive bayes are satisfied in
that sense that it is plausible that all of the categories of predictors
are independent. It could be argued that things like ``networth'',
``credit'' and ``employment'' are not independent in practice\ldots{}
but here we will assume that they if only because I'd like to answer
part 3 of the question.

We will look at the posterior probability of credit given prospect.

\begin{Shaded}
\begin{Highlighting}[]
\NormalTok{prob.employed <-}\StringTok{ }\DecValTok{7}\OperatorTok{/}\DecValTok{14}
\NormalTok{prob.unemployed <-}\StringTok{ }\DecValTok{1}\OperatorTok{-}\NormalTok{prob.employed}

\NormalTok{prob.yes.given.employed <-}\StringTok{ }\NormalTok{value[[}\DecValTok{13}\NormalTok{]]}
\NormalTok{prob.no.given.employed <-}\StringTok{ }\NormalTok{value[[}\DecValTok{14}\NormalTok{]]}
\NormalTok{prob.yes.given.unemployed <-}\StringTok{ }\NormalTok{value[[}\DecValTok{15}\NormalTok{]]}
\NormalTok{prob.no.given.unemployed <-}\StringTok{ }\NormalTok{value[[}\DecValTok{16}\NormalTok{]] }

\NormalTok{prob.employed.given.yes <-}\StringTok{ }\NormalTok{(prob.yes.given.employed }\OperatorTok{*}\StringTok{ }\NormalTok{prob.employed )}\OperatorTok{/}\NormalTok{((prob.yes.given.employed }\OperatorTok{*}\NormalTok{prob.employed) }\OperatorTok{+}\StringTok{ }\NormalTok{(prob.yes.given.unemployed }\OperatorTok{*}\StringTok{ }\NormalTok{prob.unemployed)) }

\NormalTok{prob.employed.given.no <-}\StringTok{  }\NormalTok{(prob.yes.given.unemployed }\OperatorTok{*}\StringTok{ }\NormalTok{prob.unemployed) }\OperatorTok{/}\NormalTok{((prob.yes.given.employed }\OperatorTok{*}\NormalTok{prob.employed) }\OperatorTok{+}\StringTok{ }\NormalTok{(prob.yes.given.unemployed }\OperatorTok{*}\StringTok{ }\NormalTok{prob.unemployed)) }


\NormalTok{prob.unemployed.given.yes <-}\StringTok{ }\DecValTok{1}\OperatorTok{-}\NormalTok{prob.employed.given.yes}
\NormalTok{prob.unemployed.given.no <-}\StringTok{ }\DecValTok{1}\OperatorTok{-}\NormalTok{prob.employed.given.no}

\NormalTok{e <-}\StringTok{ }\KeywordTok{c}\NormalTok{(prob.employed.given.yes,prob.employed.given.no)}
\NormalTok{u <-}\StringTok{ }\KeywordTok{c}\NormalTok{(prob.unemployed.given.yes,prob.unemployed.given.no)}

\NormalTok{out <-}\StringTok{ }\KeywordTok{cbind}\NormalTok{(e,u)}

\KeywordTok{row.names}\NormalTok{(out) <-}\StringTok{ }\KeywordTok{c}\NormalTok{(}\StringTok{"Prospect.Yes"}\NormalTok{, }\StringTok{"Prospect.No"}\NormalTok{)}
\KeywordTok{colnames}\NormalTok{(out) <-}\StringTok{ }\KeywordTok{c}\NormalTok{(}\StringTok{"Employed"}\NormalTok{,}\StringTok{"Unemployed"}\NormalTok{)}

\KeywordTok{kable}\NormalTok{(out)}
\end{Highlighting}
\end{Shaded}

\begin{longtable}[]{@{}lrr@{}}
\toprule
& Employed & Unemployed\tabularnewline
\midrule
\endhead
Prospect.Yes & 0.3333333 & 0.6666667\tabularnewline
Prospect.No & 0.6666667 & 0.3333333\tabularnewline
\bottomrule
\end{longtable}

\subsection{Question 2}\label{question-2}

You just recently joined a datascience team.

There are two datasets junk1.txt and junk2.csv They have two options 1.
They can go back to the client and ask for more data to remedy problems
with the data. 2. They can accept the data and undertake a major
analytics exercise.

The team is relying on your dsc skills to determine how they should
proceed.

Can you explore the data and recommend actions for each file enumerating
the reasons.

\subsubsection{Initial Observations}\label{initial-observations}

Upon loading the files I note that they have different extentions and
different delimiters, despite looking reasonably similar otherwise. The
inconsistencies between these 2 files would probably cause me to
approach them with a higher level of scrutiny than would otherwise be
the case.

In addition, I would likely request additional data regarding the
context of the data. What is the process that produces these data? What
info is there available about the origin of the data. This data would be
the basis for forming some intuition about the data - intuition is
essential in data science.

\begin{Shaded}
\begin{Highlighting}[]
\CommentTok{#load the data}
\NormalTok{junk1 <-}\StringTok{ }\KeywordTok{read.csv}\NormalTok{(}\StringTok{"https://raw.githubusercontent.com/plb2018/DATA622/master/homework1/data/junk1.txt"}\NormalTok{, }\DataTypeTok{header=}\NormalTok{T,}\DataTypeTok{sep =} \StringTok{" "}\NormalTok{)}
\NormalTok{junk2 <-}\StringTok{ }\KeywordTok{read.csv}\NormalTok{(}\StringTok{"https://raw.githubusercontent.com/plb2018/DATA622/master/homework1/data/junk2.csv"}\NormalTok{, }\DataTypeTok{header=}\NormalTok{T, }\DataTypeTok{sep =} \StringTok{","}\NormalTok{)}
\end{Highlighting}
\end{Shaded}

\subsubsection{Inspect the data}\label{inspect-the-data}

In both cases we have data in ``long'' format where each each row
contains a full set of observations or a ``case''. We have variables
``a'' \& ``b'' which are floating point and apparently continuous and a
``class'' variable which appears to be an integer and based on the name,
likely a categorical variable.

\begin{Shaded}
\begin{Highlighting}[]
\KeywordTok{kable}\NormalTok{(}\KeywordTok{head}\NormalTok{(junk1,}\DecValTok{5}\NormalTok{))}
\end{Highlighting}
\end{Shaded}

\begin{longtable}[]{@{}rrr@{}}
\toprule
a & b & class\tabularnewline
\midrule
\endhead
1.6204214 & 3.0036241 & 1\tabularnewline
1.4340220 & 0.7852487 & 1\tabularnewline
2.4766615 & 0.9367761 & 1\tabularnewline
0.5283093 & 0.1196222 & 1\tabularnewline
1.0054081 & 0.7872866 & 1\tabularnewline
\bottomrule
\end{longtable}

\begin{Shaded}
\begin{Highlighting}[]
\KeywordTok{kable}\NormalTok{(}\KeywordTok{head}\NormalTok{(junk2,}\DecValTok{5}\NormalTok{))}
\end{Highlighting}
\end{Shaded}

\begin{longtable}[]{@{}rrr@{}}
\toprule
a & b & class\tabularnewline
\midrule
\endhead
3.1886481 & 0.9291774 & 0\tabularnewline
0.8224527 & 0.0476031 & 0\tabularnewline
0.8147247 & 0.0291093 & 0\tabularnewline
-1.5065362 & 3.1323136 & 0\tabularnewline
0.4426887 & 2.8494282 & 0\tabularnewline
\bottomrule
\end{longtable}

\subsubsection{Look for missing cases}\label{look-for-missing-cases}

Next we check for missing values and find that there are no incomplete
cases in either of the files.

\begin{Shaded}
\begin{Highlighting}[]
\NormalTok{missing.data <-}\StringTok{ }\KeywordTok{data.frame}\NormalTok{(}\KeywordTok{c}\NormalTok{(}\KeywordTok{sum}\NormalTok{(}\OperatorTok{!}\KeywordTok{complete.cases}\NormalTok{(junk1)),}\KeywordTok{sum}\NormalTok{(}\OperatorTok{!}\KeywordTok{complete.cases}\NormalTok{(junk2))))}
\KeywordTok{row.names}\NormalTok{(missing.data) <-}\StringTok{ }\KeywordTok{c}\NormalTok{(}\StringTok{"Junk1"}\NormalTok{,}\StringTok{"Junk2"}\NormalTok{)}
\KeywordTok{colnames}\NormalTok{(missing.data) <-}\StringTok{ }\KeywordTok{c}\NormalTok{(}\StringTok{"Number of Missing Values"}\NormalTok{)}

\KeywordTok{kable}\NormalTok{(missing.data)}
\end{Highlighting}
\end{Shaded}

\begin{longtable}[]{@{}lr@{}}
\toprule
& Number of Missing Values\tabularnewline
\midrule
\endhead
Junk1 & 0\tabularnewline
Junk2 & 0\tabularnewline
\bottomrule
\end{longtable}

\subsubsection{Summary Statistics}\label{summary-statistics}

Now we compute summary statistics for reference. It may be difficult to
deduce anything from the summary stats alone without actually inspecting
the data visually, however, we compute them now so that we can refer
back to them as needed.

\begin{Shaded}
\begin{Highlighting}[]
\KeywordTok{describe}\NormalTok{(junk1)}
\end{Highlighting}
\end{Shaded}

\begin{verbatim}
##       vars   n mean   sd median trimmed  mad   min  max range skew kurtosis
## a        1 100 0.05 1.27  -0.05    0.03 1.47 -2.30 3.01  5.30 0.13    -0.80
## b        2 100 0.01 1.45  -0.07   -0.02 1.49 -3.17 3.10  6.27 0.12    -0.53
## class    3 100 1.50 0.50   1.50    1.50 0.74  1.00 2.00  1.00 0.00    -2.02
##         se
## a     0.13
## b     0.14
## class 0.05
\end{verbatim}

\begin{Shaded}
\begin{Highlighting}[]
\KeywordTok{describe}\NormalTok{(junk2)}
\end{Highlighting}
\end{Shaded}

\begin{verbatim}
##       vars    n  mean   sd median trimmed  mad   min  max range  skew kurtosis
## a        1 4000 -0.05 1.30   0.09   -0.02 1.40 -4.17 4.63  8.79 -0.17    -0.34
## b        2 4000  0.06 1.31  -0.08    0.03 1.39 -3.90 4.31  8.22  0.21    -0.35
## class    3 4000  0.06 0.24   0.00    0.00 0.00  0.00 1.00  1.00  3.61    11.06
##         se
## a     0.02
## b     0.02
## class 0.00
\end{verbatim}

\subsubsection{Visualize the Data}\label{visualize-the-data}

From the plots below we can see that:

\paragraph{Junk1}\label{junk1}

\begin{itemize}
\tightlist
\item
  Both A and B appear to be reasonably normally distributed around zero.
  B appears to be bi-modal however this is likely a resolution issue due
  to small sample size. Density plots are also shown as they better
  illustrate the shape of the distributions given the sample size.
\item
  The pairs plot shows that A and B appear to be uncorrelated
\end{itemize}

\begin{Shaded}
\begin{Highlighting}[]
\NormalTok{junk1 }\OperatorTok
\StringTok{  }\KeywordTok{gather}\NormalTok{() }\OperatorTok\StringTok{ }
\StringTok{  }\KeywordTok{ggplot}\NormalTok{(}\KeywordTok{aes}\NormalTok{(value)) }\OperatorTok{+}
\StringTok{    }\KeywordTok{facet_wrap}\NormalTok{(}\OperatorTok{~}\StringTok{ }\NormalTok{key, }\DataTypeTok{scales =} \StringTok{"free"}\NormalTok{) }\OperatorTok{+}
\StringTok{    }\KeywordTok{geom_histogram}\NormalTok{(}\DataTypeTok{bins=}\DecValTok{100}\NormalTok{)}\OperatorTok{+}
\StringTok{    }\KeywordTok{ggtitle}\NormalTok{(}\StringTok{"Junk1 - Histograms"}\NormalTok{)}
\end{Highlighting}
\end{Shaded}

\includegraphics{Data622_Homework1_files/figure-latex/unnamed-chunk-10-1.pdf}

\begin{Shaded}
\begin{Highlighting}[]
\NormalTok{junk1 }\OperatorTok
\StringTok{  }\KeywordTok{gather}\NormalTok{() }\OperatorTok\StringTok{ }
\StringTok{  }\KeywordTok{ggplot}\NormalTok{(}\KeywordTok{aes}\NormalTok{(value)) }\OperatorTok{+}
\StringTok{    }\KeywordTok{facet_wrap}\NormalTok{(}\OperatorTok{~}\StringTok{ }\NormalTok{key, }\DataTypeTok{scales =} \StringTok{"free"}\NormalTok{) }\OperatorTok{+}
\StringTok{    }\KeywordTok{geom_density}\NormalTok{()}\OperatorTok{+}
\StringTok{    }\KeywordTok{ggtitle}\NormalTok{(}\StringTok{"Junk1 - Density Plots"}\NormalTok{)}
\end{Highlighting}
\end{Shaded}

\includegraphics{Data622_Homework1_files/figure-latex/unnamed-chunk-10-2.pdf}

\begin{Shaded}
\begin{Highlighting}[]
\KeywordTok{par}\NormalTok{(}\DataTypeTok{mfrow=}\KeywordTok{c}\NormalTok{(}\DecValTok{3}\NormalTok{,}\DecValTok{1}\NormalTok{))}
\KeywordTok{plot}\NormalTok{(junk1}\OperatorTok{$}\NormalTok{a,}\DataTypeTok{type=}\StringTok{'l'}\NormalTok{)}
\KeywordTok{plot}\NormalTok{(junk1}\OperatorTok{$}\NormalTok{b,}\DataTypeTok{type=}\StringTok{'l'}\NormalTok{)}
\KeywordTok{plot}\NormalTok{(junk1}\OperatorTok{$}\NormalTok{class,}\DataTypeTok{type=}\StringTok{'l'}\NormalTok{)}
\end{Highlighting}
\end{Shaded}

\includegraphics{Data622_Homework1_files/figure-latex/unnamed-chunk-10-3.pdf}

\begin{Shaded}
\begin{Highlighting}[]
\KeywordTok{pairs}\NormalTok{(junk1, }\DataTypeTok{main=}\StringTok{"Junk1 Pairs Plot"}\NormalTok{)}
\end{Highlighting}
\end{Shaded}

\includegraphics{Data622_Homework1_files/figure-latex/unnamed-chunk-10-4.pdf}

\paragraph{Junk2}\label{junk2}

\begin{itemize}
\tightlist
\item
  Similar story between Junk1 and Junk2, however, in the distributions
  of A and B almost look like mirror images of one another.
\item
  The timeseries makes it look as if A \& B from Junk2 may be samples
  from several distinct processes. Series A has a level change at point
  1000 and a level AND variability change at point 3000. Series B has
  similar changes at point 2000 and point 3000 respectively.
\item
  If we knew more about the process it might make sense to align
  a{[}1:1000{]} and b{[}2001:3000{]}, a{[}1000:3000{]} and
  b{[}1:2000{]}, a{[}3000:end{]}, b{[}3000:end{]}
\end{itemize}

\begin{Shaded}
\begin{Highlighting}[]
\NormalTok{junk2 }\OperatorTok
\StringTok{  }\KeywordTok{gather}\NormalTok{() }\OperatorTok\StringTok{ }
\StringTok{  }\KeywordTok{ggplot}\NormalTok{(}\KeywordTok{aes}\NormalTok{(value)) }\OperatorTok{+}
\StringTok{    }\KeywordTok{facet_wrap}\NormalTok{(}\OperatorTok{~}\StringTok{ }\NormalTok{key, }\DataTypeTok{scales =} \StringTok{"free"}\NormalTok{) }\OperatorTok{+}
\StringTok{    }\KeywordTok{geom_histogram}\NormalTok{()}\OperatorTok{+}
\StringTok{    }\KeywordTok{ggtitle}\NormalTok{(}\StringTok{"Junk2 - Histograms"}\NormalTok{)}
\end{Highlighting}
\end{Shaded}

\begin{verbatim}
## `stat_bin()` using `bins = 30`. Pick better value with `binwidth`.
\end{verbatim}

\includegraphics{Data622_Homework1_files/figure-latex/unnamed-chunk-11-1.pdf}

\begin{Shaded}
\begin{Highlighting}[]
\NormalTok{junk2 }\OperatorTok
\StringTok{  }\KeywordTok{gather}\NormalTok{() }\OperatorTok\StringTok{ }
\StringTok{  }\KeywordTok{ggplot}\NormalTok{(}\KeywordTok{aes}\NormalTok{(value)) }\OperatorTok{+}
\StringTok{    }\KeywordTok{facet_wrap}\NormalTok{(}\OperatorTok{~}\StringTok{ }\NormalTok{key, }\DataTypeTok{scales =} \StringTok{"free"}\NormalTok{) }\OperatorTok{+}
\StringTok{    }\KeywordTok{geom_density}\NormalTok{()}\OperatorTok{+}
\StringTok{    }\KeywordTok{ggtitle}\NormalTok{(}\StringTok{"Junk2 - Density Plots"}\NormalTok{)}
\end{Highlighting}
\end{Shaded}

\includegraphics{Data622_Homework1_files/figure-latex/unnamed-chunk-11-2.pdf}

\begin{Shaded}
\begin{Highlighting}[]
\KeywordTok{par}\NormalTok{(}\DataTypeTok{mfrow=}\KeywordTok{c}\NormalTok{(}\DecValTok{3}\NormalTok{,}\DecValTok{1}\NormalTok{))}
\KeywordTok{plot}\NormalTok{(junk2}\OperatorTok{$}\NormalTok{a,}\DataTypeTok{type=}\StringTok{'l'}\NormalTok{)}
\KeywordTok{plot}\NormalTok{(junk2}\OperatorTok{$}\NormalTok{b,}\DataTypeTok{type=}\StringTok{'l'}\NormalTok{)}
\KeywordTok{plot}\NormalTok{(junk2}\OperatorTok{$}\NormalTok{class,}\DataTypeTok{type=}\StringTok{'l'}\NormalTok{)}
\end{Highlighting}
\end{Shaded}

\includegraphics{Data622_Homework1_files/figure-latex/unnamed-chunk-11-3.pdf}

\begin{Shaded}
\begin{Highlighting}[]
\KeywordTok{pairs}\NormalTok{(junk2, }\DataTypeTok{main=}\StringTok{"Junk2 Pairs Plot"}\NormalTok{)}
\end{Highlighting}
\end{Shaded}

\includegraphics{Data622_Homework1_files/figure-latex/unnamed-chunk-11-4.pdf}

\subsubsection{Overall Assessment}\label{overall-assessment}

I would request more information about the context and process to help
drive context about the data. Otherwise we see that

\begin{itemize}
\tightlist
\item
  The data appear to be similar to one another (junk1,junk2) and appea
  to be gaussian with a mean of apporx zero.
\item
  The data appear to be normally distributed in both cases
\item
  The data appear to be uncorrelated
\item
  The line-plot for data be shows a strange pattern which makes me
  wonder whether A and B might be reported out of synch or something
\end{itemize}

\subsubsection{Reccomendations}\label{reccomendations}

-Determine whether these two data sets can be combined (based only on
the character of the data, it appears yes) -Determine why the variance
in Junk2 isn't stable and either transform (standarize variables using
BoxCox, for example) or re-organize into more variables as appropriate

\subsection{Question 3}\label{question-3}

Read the icu.csv subset it with these 5 features in the formula and STA
is the labelcol. - Split the icu 70/30 train/test and - run the kNN.R
for K=(3,5,7,15,25,50)

Submit the result confusionMatrix, Accuracy for each K \& Plot Accuracy
vs K.

write a short summary of your findings.

\subsubsection{Load ICU}\label{load-icu}

\begin{Shaded}
\begin{Highlighting}[]
\NormalTok{icu <-}\StringTok{ }\KeywordTok{read.csv}\NormalTok{(}\StringTok{"https://raw.githubusercontent.com/plb2018/DATA622/master/homework1/data/icu.csv"}\NormalTok{, }\DataTypeTok{header=}\NormalTok{T, }\DataTypeTok{sep =} \StringTok{","}\NormalTok{)}

\CommentTok{#add coma variable as per instructions}
\NormalTok{icu}\OperatorTok{$}\NormalTok{COMA <-}\StringTok{ }\DecValTok{0}
\NormalTok{icu}\OperatorTok{$}\NormalTok{COMA[icu}\OperatorTok{$}\NormalTok{LOC }\OperatorTok{==}\StringTok{ }\DecValTok{2}\NormalTok{] <-}\DecValTok{1}

\CommentTok{#subset keeping STA as the LAST column}
\NormalTok{knn.in <-}\StringTok{ }\NormalTok{icu[}\KeywordTok{c}\NormalTok{(}\StringTok{"TYP"}\NormalTok{,}\StringTok{"COMA"}\NormalTok{,}\StringTok{"AGE"}\NormalTok{,}\StringTok{"INF"}\NormalTok{,}\StringTok{"STA"}\NormalTok{)]}
\end{Highlighting}
\end{Shaded}

\subsubsection{Load KNN.r}\label{load-knn.r}

\begin{Shaded}
\begin{Highlighting}[]
\NormalTok{euclideanDist <-}\StringTok{ }\ControlFlowTok{function}\NormalTok{(a, b)\{}
\NormalTok{  d =}\StringTok{ }\DecValTok{0}
  \ControlFlowTok{for}\NormalTok{(i }\ControlFlowTok{in} \KeywordTok{c}\NormalTok{(}\DecValTok{1}\OperatorTok{:}\NormalTok{(}\KeywordTok{length}\NormalTok{(a)) ))}
\NormalTok{  \{}
\NormalTok{    d =}\StringTok{ }\NormalTok{d }\OperatorTok{+}\StringTok{ }\NormalTok{(a[[i]]}\OperatorTok{-}\NormalTok{b[[i]])}\OperatorTok{^}\DecValTok{2}
\NormalTok{  \}}
\NormalTok{  d =}\StringTok{ }\KeywordTok{sqrt}\NormalTok{(d)}
  \KeywordTok{return}\NormalTok{(d)}
\NormalTok{\}}

\NormalTok{knn_predict2 <-}\StringTok{ }\ControlFlowTok{function}\NormalTok{(test_data, train_data, k_value, labelcol)\{}
\NormalTok{  pred <-}\StringTok{ }\KeywordTok{c}\NormalTok{()  }\CommentTok{#empty pred vector }
  \CommentTok{#LOOP-1}
  \ControlFlowTok{for}\NormalTok{(i }\ControlFlowTok{in} \KeywordTok{c}\NormalTok{(}\DecValTok{1}\OperatorTok{:}\KeywordTok{nrow}\NormalTok{(test_data)))\{   }\CommentTok{#looping over each record of test data}
\NormalTok{    eu_dist =}\KeywordTok{c}\NormalTok{()          }\CommentTok{#eu_dist & eu_char empty  vector}
\NormalTok{    eu_char =}\StringTok{ }\KeywordTok{c}\NormalTok{()}
\NormalTok{    good =}\StringTok{ }\DecValTok{0}              \CommentTok{#good & bad variable initialization with 0 value}
\NormalTok{    bad =}\StringTok{ }\DecValTok{0}
    
    \CommentTok{#LOOP-2-looping over train data }
    \ControlFlowTok{for}\NormalTok{(j }\ControlFlowTok{in} \KeywordTok{c}\NormalTok{(}\DecValTok{1}\OperatorTok{:}\KeywordTok{nrow}\NormalTok{(train_data)))\{}
 
      \CommentTok{#adding euclidean distance b/w test data point and train data to eu_dist vector}
\NormalTok{      eu_dist <-}\StringTok{ }\KeywordTok{c}\NormalTok{(eu_dist, }\KeywordTok{euclideanDist}\NormalTok{(test_data[i,}\OperatorTok{-}\KeywordTok{c}\NormalTok{(labelcol)], train_data[j,}\OperatorTok{-}\KeywordTok{c}\NormalTok{(labelcol)]))}
 
      \CommentTok{#adding class variable of training data in eu_char}
\NormalTok{      eu_char <-}\StringTok{ }\KeywordTok{c}\NormalTok{(eu_char, }\KeywordTok{as.character}\NormalTok{(train_data[j,][[labelcol]]))}
\NormalTok{    \}}
    
\NormalTok{    eu <-}\StringTok{ }\KeywordTok{data.frame}\NormalTok{(eu_char, eu_dist) }\CommentTok{#eu dataframe created with eu_char & eu_dist columns}
 
\NormalTok{    eu <-}\StringTok{ }\NormalTok{eu[}\KeywordTok{order}\NormalTok{(eu}\OperatorTok{$}\NormalTok{eu_dist),]       }\CommentTok{#sorting eu dataframe to gettop K neighbors}
\NormalTok{    eu <-}\StringTok{ }\NormalTok{eu[}\DecValTok{1}\OperatorTok{:}\NormalTok{k_value,]               }\CommentTok{#eu dataframe with top K neighbors}
 
\NormalTok{    tbl.sm.df<-}\KeywordTok{table}\NormalTok{(eu}\OperatorTok{$}\NormalTok{eu_char)}
\NormalTok{    cl_label<-}\StringTok{  }\KeywordTok{names}\NormalTok{(tbl.sm.df)[[}\KeywordTok{as.integer}\NormalTok{(}\KeywordTok{which.max}\NormalTok{(tbl.sm.df))]]}
    
\NormalTok{    pred <-}\StringTok{ }\KeywordTok{c}\NormalTok{(pred, cl_label)}
\NormalTok{    \}}
    \KeywordTok{return}\NormalTok{(pred) }\CommentTok{#return pred vector}
\NormalTok{  \}}
  

\NormalTok{accuracy <-}\StringTok{ }\ControlFlowTok{function}\NormalTok{(test_data,labelcol,predcol)\{}
\NormalTok{  correct =}\StringTok{ }\DecValTok{0}
  \ControlFlowTok{for}\NormalTok{(i }\ControlFlowTok{in} \KeywordTok{c}\NormalTok{(}\DecValTok{1}\OperatorTok{:}\KeywordTok{nrow}\NormalTok{(test_data)))\{}
    \ControlFlowTok{if}\NormalTok{(test_data[i,labelcol] }\OperatorTok{==}\StringTok{ }\NormalTok{test_data[i,predcol])\{ }
\NormalTok{      correct =}\StringTok{ }\NormalTok{correct}\OperatorTok{+}\DecValTok{1}
\NormalTok{    \}}
\NormalTok{  \}}
\NormalTok{  accu =}\StringTok{ }\NormalTok{(correct}\OperatorTok{/}\KeywordTok{nrow}\NormalTok{(test_data)) }\OperatorTok{*}\StringTok{ }\DecValTok{100}  
  \KeywordTok{return}\NormalTok{(accu)}
\NormalTok{\}}

\CommentTok{#load data}
\NormalTok{knn.df<-}\StringTok{ }\NormalTok{knn.in}
\NormalTok{labelcol <-}\StringTok{ }\DecValTok{5} \CommentTok{# for iris it is the fifth col }
\NormalTok{predictioncol<-labelcol}\OperatorTok{+}\DecValTok{1}

\CommentTok{# create train/test partitions}
\KeywordTok{set.seed}\NormalTok{(}\DecValTok{2}\NormalTok{)}
\NormalTok{n<-}\KeywordTok{nrow}\NormalTok{(knn.df)}
\NormalTok{knn.df<-}\StringTok{ }\NormalTok{knn.df[}\KeywordTok{sample}\NormalTok{(n),]}

\NormalTok{train.df <-}\StringTok{ }\NormalTok{knn.df[}\DecValTok{1}\OperatorTok{:}\KeywordTok{as.integer}\NormalTok{(}\FloatTok{0.7}\OperatorTok{*}\NormalTok{n),]}

\NormalTok{k.values <-}\StringTok{ }\KeywordTok{c}\NormalTok{(}\DecValTok{3}\NormalTok{,}\DecValTok{5}\NormalTok{,}\DecValTok{7}\NormalTok{,}\DecValTok{15}\NormalTok{,}\DecValTok{25}\NormalTok{,}\DecValTok{50}\NormalTok{)}

\NormalTok{acc <-}\StringTok{ }\KeywordTok{vector}\NormalTok{()}

\ControlFlowTok{for}\NormalTok{ (kval }\ControlFlowTok{in}\NormalTok{ k.values )\{}

\NormalTok{  K =}\StringTok{ }\NormalTok{kval }\CommentTok{# number of neighbors to determine the class}
  \KeywordTok{table}\NormalTok{(train.df[,labelcol])}
\NormalTok{  test.df <-}\StringTok{ }\NormalTok{knn.df[}\KeywordTok{as.integer}\NormalTok{(}\FloatTok{0.7}\OperatorTok{*}\NormalTok{n }\OperatorTok{+}\DecValTok{1}\NormalTok{)}\OperatorTok{:}\NormalTok{n,]}
  \KeywordTok{table}\NormalTok{(test.df[,labelcol])}
  
\NormalTok{  predictions <-}\StringTok{ }\KeywordTok{knn_predict2}\NormalTok{(test.df, train.df, K,labelcol) }\CommentTok{#calling knn_predict()}
  
\NormalTok{  test.df[,predictioncol] <-}\StringTok{ }\NormalTok{predictions }\CommentTok{#Adding predictions in test data as 7th column}
  \KeywordTok{print}\NormalTok{(}\KeywordTok{paste0}\NormalTok{(}\StringTok{"The accuracy for K="}\NormalTok{,K,}\StringTok{" is "}\NormalTok{,}\KeywordTok{accuracy}\NormalTok{(test.df,labelcol,predictioncol)))}
  \KeywordTok{print}\NormalTok{(}\StringTok{"The Confustion Matrix is:"}\NormalTok{)}
  \KeywordTok{print}\NormalTok{(}\KeywordTok{table}\NormalTok{(test.df[[predictioncol]],test.df[[labelcol]]))}
  
\NormalTok{  acc <-}\StringTok{ }\KeywordTok{c}\NormalTok{(acc,}\KeywordTok{accuracy}\NormalTok{(test.df,labelcol,predictioncol))}

\NormalTok{\}}
\end{Highlighting}
\end{Shaded}

\begin{verbatim}
## [1] "The accuracy for K=3 is 80"
## [1] "The Confustion Matrix is:"
##    
##      0  1
##   0 48  7
##   1  5  0
## [1] "The accuracy for K=5 is 85"
## [1] "The Confustion Matrix is:"
##    
##      0  1
##   0 50  6
##   1  3  1
## [1] "The accuracy for K=7 is 83.3333333333333"
## [1] "The Confustion Matrix is:"
##    
##      0  1
##   0 50  7
##   1  3  0
## [1] "The accuracy for K=15 is 88.3333333333333"
## [1] "The Confustion Matrix is:"
##    
##      0  1
##   0 53  7
## [1] "The accuracy for K=25 is 88.3333333333333"
## [1] "The Confustion Matrix is:"
##    
##      0  1
##   0 53  7
## [1] "The accuracy for K=50 is 88.3333333333333"
## [1] "The Confustion Matrix is:"
##    
##      0  1
##   0 53  7
\end{verbatim}

\begin{Shaded}
\begin{Highlighting}[]
\KeywordTok{plot}\NormalTok{(acc,}\DataTypeTok{main=}\StringTok{"Accuracy Plot"}\NormalTok{,}
        \DataTypeTok{xlab=}\StringTok{"Value of K"}\NormalTok{,}
        \DataTypeTok{ylab=}\StringTok{"Accuracy"}\NormalTok{,}
        \DataTypeTok{xaxt=}\StringTok{"n"}\NormalTok{)}



\KeywordTok{axis}\NormalTok{(}\DecValTok{1}\NormalTok{,}\DataTypeTok{at=}\DecValTok{1}\OperatorTok{:}\DecValTok{6}\NormalTok{,}\DataTypeTok{labels =}\NormalTok{ k.values)}
\end{Highlighting}
\end{Shaded}

\includegraphics{Data622_Homework1_files/figure-latex/unnamed-chunk-13-1.pdf}

\subsubsection{Summary}\label{summary}

Here I was able to successfully run the code as per the instructions. In
this case (i.e.~ICU data) we can see that accuracy continues to climb
intul it reaches 15, where it remains steady.

\end{document}
